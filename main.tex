\documentclass[conference]{IEEEtran}


%XXX Remove before final submission
\usepackage{todonotes}
\usepackage{cite}
%\usepackage{amsmath}
%
% Note that the amsmath package sets \interdisplaylinepenalty to 10000
% thus preventing page breaks from occurring within multiline equations. Use:
%\interdisplaylinepenalty=2500
% after loading amsmath to restore such page breaks as IEEEtran.cls normally
\usepackage{url}

% correct bad hyphenation here
\hyphenation{}


\begin{document}

%XXX remove before submission
\newcommand{\TODO}[1]{\todo[inline]{#1}}
\newcommand{\TODOFIG}[1]{\missingfigure[figheight=\columnwidth]{#1}}

\title{The Statistics of Benchmarks for Testing Perfomance Regressions}

% author names and affiliations
% use a multiple column layout for up to three different
% affiliations
\author{\IEEEauthorblockN{Jiahao Chen and Jarrett Revels}
\IEEEauthorblockA{Computer Science and Artificial Intelligence Laboratory\\
Massachusetts Institute of Technology\\
Cambridge, Massachusetts 02139--4307\\
Email: \{jiahao,jrevels\}@csail.mit.edu}
}

% make the title area
\maketitle

\begin{abstract}
\TODO{The abstract goes here.}
\end{abstract}

\IEEEpeerreviewmaketitle

\section{Introduction}

If we want to be able to say that ``Program A is faster than Program B'', we
need to time both programs.

Modern computer systems make benchmarking hard. They do many things under the
hood that can screw up the comparison~\cite{HP5e}.

\subsection{Factors affecting timings}

\TODO{jrevels to fill in}

\label{sec:statmodel}
\section{A simple statistical model}

\TODO{jiahao to fill in}

\begin{figure}[!t]
\centering
\TODOFIG{Ideal benchmark timings}
%\includegraphics[width=2.5in]{myfigure}
\caption{How benchmark timings ought to scale with the number of repetitions $n$ according to the statistical model of Sec.~\ref{sec:statmodel}.}
\label{fig:idealscaling}
\end{figure}

\section{Results on some Julia benchmarks}

\begin{figure}[!t]
\centering
\TODOFIG{Some clusters}
%\includegraphics[width=2.5in]{myfigure}
\caption{Clustering of Julia benchmarks.}
\label{fig:benchclusters}
\end{figure}

\begin{figure}[!t]
\centering
\TODOFIG{Real benchmark timings}
%\includegraphics[width=2.5in]{myfigure}
\caption{How benchmark timings scale with the number of repetitions $n$.}
\label{fig:scaling}
\end{figure}

% An example of a floating table. Note that, for IEEE style tables, the
% \caption command should come BEFORE the table and, given that table
% captions serve much like titles, are usually capitalized except for words
% such as a, an, and, as, at, but, by, for, in, nor, of, on, or, the, to
% and up, which are usually not capitalized unless they are the first or
% last word of the caption. Table text will default to \footnotesize as
% the IEEE normally uses this smaller font for tables.
% The \label must come after \caption as always.
%
%\begin{table}[!t]
%% increase table row spacing, adjust to taste
%\renewcommand{\arraystretch}{1.3}
% if using array.sty, it might be a good idea to tweak the value of
% \extrarowheight as needed to properly center the text within the cells
%\caption{An Example of a Table}
%\label{table_example}
%\centering
%% Some packages, such as MDW tools, offer better commands for making tables
%% than the plain LaTeX2e tabular which is used here.
%\begin{tabular}{|c||c|}
%\hline
%One & Two\\
%\hline
%Three & Four\\
%\hline
%\end{tabular}
%\end{table}

\section{Conclusion}

\TODO{The conclusion goes here.}

\section*{Acknowledgment}

We thank the many Julia developers for many insightful discussions.

This work was supported by the Nanosoldier grant.\TODO{Get grant info}

\bibliography{biblio}
\bibliographystyle{IEEEtran}

% that's all folks
\end{document}

